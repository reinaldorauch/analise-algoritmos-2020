\documentclass[12pt]{article}

\usepackage{sbc-template}
\usepackage{graphicx,url}
\usepackage[utf8]{inputenc}
\usepackage[brazil]{babel}
\usepackage{fancyvrb}

\newcommand\slsh{\char`\\}

\usepackage{hyperref}
\hypersetup{
    colorlinks=true,
    linkcolor=cyan,
    filecolor=blue,
    urlcolor=cyan,
    citecolor=cyan,
}

\sloppy

\title{Análise Experimental entre os algoritmos Quicksort, Radixsort  e Mergesort}

\author{Reinaldo Antonio Camargo Rauch\inst{1}}


\address{Universidade Estadual de Ponta Grossa (UEPG)}

\begin{document}

\maketitle

\begin{resumo}
Este relatório visa comparar o tempo de execução entre os algoritmos Quicksort, Radixsort (Ordenação Digital) e Mergesort.
\end{resumo}

\section{Introdução}

Será utilizado uma máquina com as especificações:

\begin{itemize}
  \item Processador i7-4790K @ 4 GHz
  \item Placa Mãe Gigabyte GA-H97M-Gaming 3 (rev 1.0)
  \item Memória RAM: 16GB
  \item Sistema Operacional: MacOS Catalina 10.15.7 (Não oficial)
\end{itemize}


Ferramentas de compilação:

\begin{itemize}
  \item XCode 12.4
  \item Clang 12.0.0
\end{itemize}

\section{Algoritmos}

Os algoritmos que serão comparados são Quicksort, Radixsort e o Mergesort.

\subsection{Quicksort}

O Quicksort é um algoritmo de divisão e conquista, que encontra o pivô com o método de encontrar a mediana dos valores, de modo a evitar o comportamento $\theta{(n^2)}$, segundo a implementação utilizada na stdlib da linguagem de programação C, a qual os testes serão implementados. Esta implementação é baseada no algoritmo de Hoare de 1962 \cite{quicksort:21}.

\subsection{Radixsort ou Ordenação Digital}

O Radixsort ou Ordenação Digital, é um algoritmo que funciona  \cite{radixsort:21}.

\subsection{Mergesort}

O Mergesort utilizado é a implementação da stdlib do C \cite{radixsort:21}

\section{Experimento}

O experimento foi modelado para rodar com multithreading mas foi detectado que essa configuração afeta a consistência da execução dos tempos dos testes então foi fixado para utilizar somente uma thread de execução.

A variação do N dos expermentos foi definida para iniciar em 100 e ir até 300000 com incrementos também de 100 pois tem uma precisão suficiente para demonstrar o comportamento do tempo de execução dos algoritmos.

\section{Resultados}

\section{Conclusão}

Conclusão.

\bibliographystyle{sbc}
\bibliography{referencias.bib}

\section*{Anexo}

\end{document}
