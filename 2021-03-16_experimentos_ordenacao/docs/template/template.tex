\documentclass[12pt]{article}

\usepackage{sbc-template}
\usepackage{graphicx,url}
\usepackage[utf8]{inputenc}
\usepackage[brazil]{babel}
\usepackage{fancyvrb}

\newcommand\slsh{\char`\\}

\usepackage{hyperref}
\hypersetup{
    colorlinks=true,
    linkcolor=cyan,
    filecolor=blue,      
    urlcolor=cyan,
    citecolor=cyan,
}
 
\sloppy

\title{Análise Experimental entre os algoritmos Quicksort, Radixsort (Ordenação Digital e }

\author{Reinaldo Antonio Camargo Rauch\inst{1}}


\address{Universidade Estadual de Ponta Grossa (UEPG)}

\begin{document} 

\maketitle
     
\begin{resumo} 
  Este relatório visa comparar o tempo de execução entre os algoritmos Quicksort, Radixsort (Ordenação Digital) e Mergesort. Descobrimos que mesmo o Mergesort tendo um 
\end{resumo}

\section{Introdução}

Texto da Introdução.

\section{Primeira Seção}

Texto da primeira Seção. Exemplo de citação \cite{knuth:84}.

\subsection{Uma Subseção}

\subsection{Outra Subseção}

\section{Segunda Seção}

Texto da segunda seção do trabalho.

\section{Conclusão}

Conclusão. 

\bibliographystyle{sbc}
\bibliography{referencias.bib}

\section*{Anexo}


\end{document}
